\documentclass[12pt]{article}

% set margins and spacing
\addtolength{\textwidth}{1.3in}
\addtolength{\oddsidemargin}{-.65in} %left margin
\addtolength{\evensidemargin}{-.65in}
\setlength{\textheight}{9in}
\setlength{\topmargin}{-.5in}
\setlength{\headheight}{0.0in}
\setlength{\footskip}{.375in}
\renewcommand{\baselinestretch}{1.0}
\linespread{1.0}

% load miscellaneous packages
\usepackage{csquotes}
\usepackage[american]{babel}
\usepackage[usenames,dvipsnames]{color}
\usepackage{graphicx,amsbsy,amssymb, amsmath, amsthm, MnSymbol,bbding,times, verbatim,bm,pifont,pdfsync,setspace,natbib}

% enable hyperlinks and table of contents
\usepackage[pdftex,
bookmarks=true,
bookmarksnumbered=false,
pdfview=fitH,
bookmarksopen=true,hyperfootnotes=false]{hyperref}

% define environments
\newtheorem{definition}{Definition}
\newtheorem{fact}{Fact}
\newtheorem{result}{Result}
\newtheorem{proposition}{Proposition}



\begin{document}
\title{Working Title: Nepobabies: Nepotitstic Behavior and Labor Market Entry}
\author{Rachel Rabinowitz\thanks{Syracuse University, Economics Department. Email: kbuzard@syr.edu.} \and Ryan Seely\thanks{}}
\date{\vskip-.1in \today}
\maketitle

\vskip.3in
\begin{center} {\bf Abstract} \end{center}

\begin{quote}
{\small Hi Professor Buzard! This is the working draft for the research report we have been working. We request that you provide feedback on the literature review and data sections at your convenience, as those sections, outside of the results, are the only ones we are ready to present.}
\end{quote}

%\iffalse makes the section a comment
\iffalse
\bigskip
\section{Introduction} \label{sec:introduction}

Answer the questions
\begin{enumerate}
    \item \textbf{Why should the reader care? / Why is the topic important?} (required)
    \item Why did you choose this topic? (optional)
    \item \textbf{What question will you answer? How will you do it?} (required)
        \begin{enumerate}
            \item If your theory/hypothesis fit in one paragraph, include it here. If it is longer, make it a separate section after the lit review. EITHER OPTION IS FINE as long as the length is sufficient/appropriate for your project.
        \end{enumerate}
    \item \textbf{What did you find?} (required)
    \item \textbf{Give a "road map" of the paper. Where will the reader find the various parts of your work?} (required)
\end{enumerate}
\fi
%\fi ends the comment - remove this to bring back the intro

\section{Literature Review} \label{sec:literature}

\textbf{Kahn, L. B. (2010). The long-term labor market consequences of graduating from college in a bad economy. Labour economics, 17(2), 303-316.}

    The effects of graduating college in a poor economy can have many negative effects throughout your lifetime. While analyzing these effects, Khan examined the relationship between economic conditions at the time of graduation and your ability to find work for the first two decades after graduation. Khan’s theory was that those graduating in a worse-off economy will experience higher unemployment as there are fewer jobs available and less demand for work from firms. Those graduating in worse economic conditions also would be assumed to have a greater likelihood of job mismatching. When graduating in poor economic conditions, those experiencing difficulties in entering the workforce may find themselves returning to education, to increase their human capital and enter the workforce during a better economic cycle where there is more demand and opportunity. Without returning to education, those graduating in poor economic conditions can be expected to follow in lower-level occupations, but have longer tenure at their firms with the assumption that they can grow within the company. Those graduating in a worse-off economy can be expected to experience long-term effects throughout their lifetime. 


  \textbf{Schwandt, H., & Von Wachter, T. (2019). Unlucky cohorts: Estimating the long-term effects of entering the labor market in a recession in large cross-sectional data sets. Journal of Labor Economics, 37(S1), S161-S198.}
	
    Recessions are seen to be impactful to recent graduates, but the impacts on minority groups are seen to be exponential. In poor economic conditions, there is less demand for labor and low-skilled workers, so those entering the workforce are more likely to experience lower earnings and wage reductions. The fewer jobs demanded in a competitive labor market will be more favorable to high-skilled workers, impacting the ability of recent graduates to find work after graduation. When entering the labor market during a recession, recent graduates can be expected to experience more unemployment as they do not possess a higher level of education that firms cater to during recessions. 

 \textbf{ Hellerstein, J. K., & Morrill, M. S. (2011). Dads and Daughters: The Changing Impact of Fathers on Women’s Occupational Choices. The Journal of Human Resources, 46(2), 333–372. http://www.jstor.org/stable/41304823}
 
	Young females have become increasingly more likely to enter the same profession as their fathers. As young females become more likely to enter male-dominated occupations, daughters are likely to go into their father’s occupations and follow in the family's footsteps. Since there may be a sense of inside knowledge and/or networking, daughters may expect to have a better likelihood of finding a job after graduation, especially during worse economic times. As women are naturally more risk-averse than men, this can be seen as a successful path, as their father can guide them. On the contrary, sons haven’t been increasingly going into their father's occupations as this number has consistently stayed constant whereas women entering male-dominated occupations have begun to increase within the past decades. If more women are entering these male-dominated fields, there is more scope to follow their fathers, leading to a higher correlation between daughters following their father’s professions. 


\textbf{ Oreopoulos, P., Von Wachter, T., & Heisz, A. (2012). The short- and long-term career effects of graduating in a recession. American Economic Journal: Applied Economics, 4(1), 1-29.}
	
    During recessions, higher-skilled workers may take lower-skilled jobs to avoid unemployment, resulting in the quality of jobs declining in the labor market. The ability to differentiate yourself after graduation may be a helpful tool in order to gain employment in a tight labor market. Whether at a higher education level, through college prestige, or on-site training, graduates can differentiate themselves to recover these losses faced when entering the workforce during recessions. Those with higher education or training are more likely to move quickly throughout firms as they are in higher demand than lower-skilled workers, especially in poor economic conditions. Those seeking employment are more likely to accept lower-skilled jobs as they are in much higher demand during poor economic conditions.

 \textbf{Kramarz, F., & Skans, O. N. (2014). When strong ties are strong: Networks and youth labour market entry. Review of Economic Studies, 81(3), 1164-1200.}
	
    Strong family ties are an important determinant for where graduates find their first stable job. Employees are more likely to disseminate information about job openings to their strong ties first due to more table networks and greater maintenance given to create those networks. The information on job vacancies will be given to those with stronger network ties first, and then to weaker networks, resulting in those with stronger ties having a higher likelihood of receiving employment. Firms recruit more often from social connections of high-quality incumbent workers, as firms believe them to inherit their qualities, especially when ex-ante information about the applicant is poor. Firms also benefit from recruiting the children of incumbent employees due to the relative wage stagnation of the parent employees and the signaled quality of the recruited workers. Stronger family ties are seen to be helpful, especially when finding employment during tight economic times. 

\iffalse
\section{Theoretical Analysis}
\label{sec:theory}
Optional--may include in intro if it's short.
\fi

\section{Data}
\label{sec:data}
We use data collected from the GSS in the years 2002, 2006, 2010, 2014, and 2018. The surveys are largely conducted through 90-minute in-person interviews and are done with English and Spanish-speaking adults to create a nationally representative sample. Only observations from the aforementioned years were analyzed due to a key variable, which asks respondents the length of time they have occupied their current job, being present in these five survey years. Using this variable, we analyzed the age at which someone was hired. Using the respondent's age and this key variable measuring job length, we were able to estimate the age at which respondents were hired and further limited our sample to young adults, which we defined as younger than 30 years old. After limiting our sample to young adults and the years where the job length variable was present, there were 3,560 observations and 288 nepobabies, defined as young adults who were in the same industry as their parents. 

The GSS does not include the monthly unemployment rates which we used to analyze the difference in hiring patterns of nepobabies. So, we used data from the Bureau of Labor Statistics to assign monthly unemployment rate values to the months that respondents were hired to measure the competitiveness of the labor market at the time of hiring. This unemployment rate data spanned from January of 1987 to December of 2018 to include the earliest month and year that a young adult was hired up to the end of our observations.


\section{Results}
\label{sec:result}

In the first part of our analysis, we attempt to show how nepotistic behavior changes when labor markets become more competitive. In the second part, we analyze the demographics and parental relationships of nepobabies.


First, we defined nepobabies as having a matching industry code as at least one parent. Separately, we were able to estimate the month and year that a respondent was hired by using their response to the question asking the length of time they have held their job. Using the BLS data that was assigned to the month that respondents were hired, we could then see the labor market conditions under which respondents were hired. We created three groups of nepobabies, hired during low unemployment, hired during medium unemployment, and hired during high unemployment. In our analysis, low unemployment is considered to be an unemployment rate of less than 4.9 percent. High unemployment is considered to be greater than 6.6 percent, and average employment is the space in between those unemployment rates. We determined these to be the cutoff for high and low unemployment rates because they are the first and third quartiles of all unemployment rates in our sample of hire months. We decided to measure the difference in means between the top and bottom quarter of all the unemployment rates in our sample because we found that when only dividing the sample into two parts and setting the cutoff unemployment rate at some measure of central tendency between five and six percent, we would get very different results from our t-test. We cut out the middle 50 percent of months, i.e. medium unemployment rate months, in order to reduce the noise of the responses close to the average and only look at the extremes.


After cutting up the sample in this manner, we conducted a two-sample t-test measuring the statistical significance of the means between nepobabies hired in unemployment and nepobabies hired in low unemployment. Our null hypothesis was that there was no significant difference between the mean of nepobabies hired during high unemployment and the mean of nepobabies hired during low unemployment. The result from the t-test gives us enough evidence to reject the null hypothesis with 99 percent confidence, where p <  0.01. From this test we find that approximately 24 percent of the sample’s nepobabies were hired during low unemployment and that 36 percent of the sample’s nepobabies were hired during times of high unemployment. This provides strong evidence that young adults rely on and use the support and connections of their parents to find jobs more so when the labor market is more competitive. Additionally, the confidence intervals for the two means have no overlap, providing further evidence for the idea that children are more likely to join their parents’ industries during bad times. We then analyzed how the different unemployment rates, as representations for level of competition in the labor market, were related to the rest of our sample’s hire date. We found the ratio of nepobabies hired in high unemployment to non-nepobabies hired in high unemployment, which is approximately 0.11. That same ratio, but for nepobabies and non-nepobabies hired in low unemployment, was nearly half that at 0.065. What we can extrapolate from that is that nepobabies are much more likely to rely on their parent’s for a job during highly competitive labor markets than during times with low labor market competition.


For the second part of our analysis, we looked at the gender dynamics of nepotistic parent-child relationships and the demographics of nepobabies. As one could imagine, we find that children are much more likely to find themselves in a nepotistic relationship with their parent of the same gender than they are for the opposite gender parent. Nepotistic mothers, of which there were 119 when excluding mothers who had the same industry as the respondent’s father, were much more likely to have their daughters in the same industry as their sons. Of the nepotistic mother-child relationships, 71 percent were daughters and only 29 percent were sons. This highly gendered relationship holds true for nepotistic fathers. When excluding fathers who had the same industry as the respondent’s mother, there were 147 nepotistic fathers. Of the nepotistic father-child relationships, 80 percent were sons and only 20 percent were daughters. 


We then examined different demographics, as recorded in the GSS observations, to compare nepobabies to the rest of the sample. First, we tested how race differs between the sample and the nepobaby population within our sample. The sample is 72.3 percent white and the nepobabies are 77.4 percent white. After running a two-sample t-test to measure the significance of the difference of means between these two populations, we find that the difference is statistically significant, and can say with 95 percent confidence that nepobabies are more likely to be white than their counterparts whose parents are in different industries. We also examined how nepobabies may feel differently in the workplace than their counterparts. Using a variable that marks the likelihood that a respondent thinks they will lose their job, we found that nepobabies feel more secure in their workplaces. While only 65 percent of all workers in our sample said they think it is not likely that they will lose their job, 73 percent of nepobabies believe that it is unlikely they will lose their job. This provides further evidence of the benefit that nepobabies receive by taking advantage of their parental connections to find and maintain employment.
We ran similar two-sample t-tests for the following demographics and found no statistically significant difference between the nepobaby sample and the whole sample: gender, income, class (self-reported), and the amount of hours worked per week. 
 


\iffalse
\section{Discussion}
\label{sec:discussion}

Optional. This is where you would discuss any of the following
\begin{itemize}
    \item caveats (are there problems with the data that there are no obvious ways to resolve? if so, how might this impact
    \item future work / next steps
    \item implications of the results (how your findings -- if they were causally identified -- might inform policymaking, etc.
\end{itemize}

\section{Conclusion}
\label{sec:conclusion}

Re-state (in different words) what you did and what you learned. If your discussion would be short, you can add the discussion after your summary.
\fi

\newpage
\section*{Bibliography}
\singlespacing
\setlength\bibsep{0pt}


\newpage
\section*{Appendix A. Placeholder} \label{sec:appendixa}
\addcontentsline{toc}{section}{Appendix A}

\end{document}